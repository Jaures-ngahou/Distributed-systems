% % % % % % % % % % % % % % % % % % % % % % % % % % % % % % % % % % % % % % % % % % 
% Exemple de rapport de master
% Version originale :  A. Tarantola, Octobre 2004
% Version modifiée :   A. Fournier et G. Moguilny, Mars 2012...Novembre 2022
% % % % % % % % % % % % % % % % % % % % % % % % % % % % % % % % % % % % % % % % % % 

\documentclass{ipgpmaster}
%
% Pour saisir directement les lettres accentuées :
% \usepackage[applemac]{inputenc} % pour utilisateurs mac
% \usepackage[ansinew]{inputenc}  % pour Windows ANSI
% \usepackage[latin1]{inputenc}   % pour Unix Latin1
\usepackage[T1]{fontenc}
\usepackage{amsmath}


\begin{document}
\checkyears

\vspace*{5mm}

%\setkeys{Gin}{draft} % images non affichées
\setkeys{Gin}{draft=false} % images affichées

%\linenumbers % Numerotation des lignes pour relecture
%Select here the language 
\selectlanguage{french}   
%\selectlanguage{english} 


\def\author{KENFACK NGAHOU Jaures}
\def\title{SYSTÈMES DISTRIBUÉS}
\def\shorttitle{systèmes distribués}
\def\unit{INF4218}
\def\supervisor{Dr  ADAMOU HAMZA \& GBETNKOM NJIFON Jeff }
\def\keywords{Sciences de la Terre,  master, rapport, conseils}

\Entete



\tableofcontents
\newpage

\section{Introduction}

Le stage de recherche du master permet aux étudiants d'apprendre le métier 
de chercheur, en participant au travail de recherche d'un laboratoire. 
Le travail d'un chercheur ressemble beaucoup à celui d'un journaliste : 
il faut d'abord avoir l'idée d'un sujet d'enquête, ensuite, il faut que 
l'enquête fournisse des informations et, enfin, il faut la publier. 
Une grande quantité d'enquêtes journalistiques, et de recherches 
scientifiques, sont restées ignorées parce que leur publication a été 
bâclée.

Rédiger un rapport scientifique prend beaucoup de temps. Après un 
``premier jet'', il faut souvent tout réécrire. 
Lorsque la forme du rapport semble acceptable, il faut demander l'avis 
de ses collègues, y apporter des modifications (souvent dramatiques), 
et itérer le processus
jusqu'à ce qu'un document de lecture facile ait été produit. Il~existe 
un \emph{style littéraire\/} très strict que les jeunes chercheurs ont parfois 
du mal à accepter. Si l'originalité est indispensable pour le sujet de 
recherche, elle est simplement distrayante dans la publication,
où on ne demande que de la perfection dans la pré\-sen\-ta\-tion.


\section{Avantages et inconvenients}


\subsection{Architectures logicielles}
\subsubsection{en couche}
Partie très importante du rapport.
Il doit permettre de se faire une idée précise des résultats.
Le style est identique à celui de l'article, simplement en résumé.
Oubliez que le résumé est publié au début de l'article et rédigez 
comme s'il devait être la seule chose qui existe : ne faites pas 
référence à l'article (si vous disiez ``nous montrons que\dots'' 
vous seriez en train de le faire). 
Ne parlez pas de ``certaines'' hypothèses ou résultats ; dites lesquels 
ou lesquelles (ou n'en parlez pas si ce n'est pas important). Il est difficile de véhiculer beaucoup 
d'information en peu de lignes, mais c'est indispensable.
\subsubsection{SOA}
\subsubsection{PubSub}

\subsection{Systèmes}
\subsubsection{Centralisés}
\subsubsection{Décentralisés}
\subsubsection{Hibride}

Sert à situer le contexte du travail. Pourquoi est-ce intéressant ? 
Qui a travaillé avant sur ce sujet? Quelle démarche avons-nous suivi ?
Il faut aussi dans l'introduction expliquer l'organisation de l'article 
(après quelques rappels de la théorie de base,  l'instrument 
utilisé est décrit dans la section 3.2\dots), sans que cela 
devienne une ``table des matières''.




Évidemment, la partie la plus importante dans le fond, très difficile 
à écrire. C'est ici que la démarche scientifique doit être 
scrupuleusement suivie. De la rigueur, de la rigueur, et de la clarté.




\section{Architecture de notre logiciel}

\section{Design goals}
\subsection{Transparency}
\subsection{Openness}
\subsection{Dependability}
\subsection{Security}
\subsection{Scalability}

\section{Conclusion}

Écrire un rapport de stage de master n'est pas aisé. 
Cela prend du temps, et les difficultés sont souvent sous-estimées.
Mais cette rédaction est une composante essentielle du stage de recherche.

Le \emph{style\/} d'un rapport technique est bien codifié, et laisse peu de place
à l'originalité, qui, par contre, aura libre cours dans le choix du sujet
ou des méthodes de recherche. On demande une grande \emph{ clarté\/} et une grande 
\emph{concision\/} dans la présentation. 
Pour cela, il faut s'y prendre avec du temps : un document écrit se travaille 
beaucoup avant de le divulguer.

 
\newpage

\addcontentsline{toc}{section}{Références}
\bibliography{master_AT}
%\nocite{*} % pour afficher toutes les entrées (même celles non citées)
            % de la base de données bibliographique
            

\cleardoublepage
  
\end{document}

